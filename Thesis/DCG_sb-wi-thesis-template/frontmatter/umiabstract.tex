\textbf{Abstract}

The 26S proteasome is the central proteolytic effector in the ubiquitin system that is responsible for degrading numerous regulators following their selective ubiquitylation.
While much is known about the construction of the yeast and mammal particles, little is known about the pathways used to assemble the plant particle.
One challenge is that the known yeast chaperones appear sufficiently diverged to preclude high-confidence identification of their plant counterparts by genomic searches.
Here, we used in-depth mass spectrometric analysis of Arabidopsis 26S proteasomes, which were affinity purified from seedlings under conditions that promote the accumulation of assembly intermediates, to identify a large collection of interacting proteins that associate with either the core protease (CP) or regulatory particle (RP).
Sequence comparisons, Y2H and BiFC studies revealed that some are likely assembly chaperones, with several CP factors harboring the signature C-terminal HbYX motif that allows their association with the $\alpha$-subunit ring.
Several of the RP-specific factors appear to be orthologs of the chaperones Nas2, Nas6, Hsm3 and Ecm29.
Whereas yeast assembles only a single particle type, mammals can assemble alternate proteasomes by replacing individual subunits with distinct isoforms (e.g., immunoproteasomes).
In plants, most 26S proteasome subunits are encoded by paralogous genes with sufficient divergence to suggest that plants also accumulate a collection of particles.
However, proteomic analysis of proteasomes selectively enriched using paralog-specific tags strongly imply that although plants possess this genetic diversity, the incorporation of these paralogs appears random, and is mainly influenced by the differential expression of the corresponding genes.
Taken together, these proteomic studies provide the first insights into plant proteasome assembly and diversity, and identify factors that build the CP and RP subcomplexes and finally the 26S holo-particle. 
