\textbf{Abstract}

The 26S proteasome is the central proteolytic effector in the ubiquitin system that is responsible for degrading numerous regulators following their selective ubiquitylation. The 26S proteasome is made up of two distinct subcomplexes, the Regulatory Particle (RP), responsible for recognizing and unfolding ubiquitylated substrates, and the Core Protease (CP), responsible for degrading the unfolded polypeptide. 
Herein, I developed an affinity purification of the 26S proteasome that targeted the RP via two isoforms of the RPT4 subunit. With this newly developed affinity purification, in conjunction with a preivously developed CP-based affinity purification of 26S proteasome, I was able to selectively enrich for either the RP or CP subcomplexes.
Mass spectrometric analyses of these affinity purified complexes identify a large collection of interacting proteins that associate with either the core protease (CP) or regulatory particle (RP).
While much is known about the construction of the yeast and mammalian particles, little is known about the pathways used to assemble the plant particle.
Sequence comparisons, Y2H and BiFC studies revealed that some proteasome associated proteins are likely assembly chaperones, with several CP factors harboring a signature C-terminal HbYX motif that typically enables association with the $\alpha$-subunit ring.
Several of the RP-specific factors appear to be orthologs of the chaperones Nas2, Nas6, Hsm3.
Whereas yeast assembles only a single particle type, mammals can assemble alternate proteasomes isotypes by replacing individual subunits with distinct isoforms (e.g., immunoproteasomes).
In plants, most 26S proteasome subunits are encoded by paralogous genes with sufficient divergence to suggest that plants also accumulate a collection of particles.
However, proteomic analysis of proteasomes selectively enriched using paralog-specific tags strongly imply that although plants possess this genetic diversity, the incorporation of these paralogs appears random.
Taken together, these proteomic studies provide the first insights into plant proteasome assembly and diversity, and identify factors that may build the CP and RP subcomplexes and finally the 26S holo-particle.